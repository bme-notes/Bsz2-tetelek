\section{4. tétel: Hamilton és Euler körök, illetve utak}

\begin{definicio}{EULER-ÚT ÉS KÖR}
A G gráf \textbf{Euler-körének} nevezünk egy zárt élsorozatot, ha az élsorozat pontosan egyszer tartalmazza G összes élét. Ha az élsorozat nem feltétlenül zárt, akkor \textbf{Euler-útról} beszélünk.
\end{definicio}

\begin{tetel}{EULER-KÖR LÉTEZÉSE}
G összefüggő gráfban akkor és csak akkor létezik Euler-kör, ha G minden pontjának fokszáma páros.
\end{tetel}

\begin{bizonyitas}{}
  Először lássuk be, hogy ha van a gráfban E-kör, akkor minden pont foka páros. Induljunk el a gráf tetszőleges pontjából és járjuk körbe az E-kör mentén. Minden pontban ugyanannyiszor mentünk be, mint ahányszor kimentünk, a ki- és bemenések száma a pont fokszáma. Ez biztosan páros.
Most lássuk, be, hogy ha $G$ összefüggő és csupa páros fokszámú csúcsa van, akkor létezik benne Euler-kör. A bizonyításához vegyük ezt az algoritmust:
\begin{itemize}
\item[\textbf{0.}] Legyen $v \in V(G)$ a gráf egy teszőleges csúcsa
\item[\textbf{1.}] Induljunk el az $v$ csúcsból élismétlés nélküli élsorozaton elakadásig, ezt az élsorozatot nevezzük $P$-nek. Ekkor belátható, hogy $P$-nek $v$-ben kell végződnie. Egyrészt nem tudunk köztes pontban elakadni, mert minden csúcs amibe bemegyünk, abból ki is (páros fokszámokat itt használjuk ki). Másrészt speciálisan a kezdő ($v$) csúcsból kifelé és befelé haladás a kör elejét és végét adja.
\item[\textbf{2.}] HA $P$ már Euler-kör $\implies$ \textbf{STOP}
\item[\textbf{3.}] HA nem akkor, létezik olyan $w$ csúcs a $P$ élsorozatban, amely $P$-beli és nem $P$-beli éleket is tartalmaz (ha nem így lenne akkor nem összefüggő a gráf). Vegyük ezt a $w $ csúcsot futassuk le rá az algoritmust ezúttal $G$ gráf $E(G) \textbackslash E(P)$ élhalmazzal rendelkező részgráfjában, ez ad egy $Q$ utat.
\item[\textbf{4.}] Adjuk össze a két élsorozatot: $v \rightarrow w + Q + w \rightarrow v$ ezt nevezzük P' élsorozatnak. Ugrás vissza a \textbf{2.} lépéshez (hiszen lehet még $w$-hez hasonló csúcs az eredeti élsorozatban).
\end{itemize}
Ezt a rekurzív algoritmust tetszőleges csupa páros fokszámú összefüggő gráfra alkalmazva végül Euler-kört kell kapjunk. Ennek belátásához, már csak az algoritmus lépésszámának végessége kell. Ami nyilvánvaló hisz $|E(P')| > |E(P)|$ (P' hosszabb élsorozat kell legyen minimum 2 éllel), ez viszont nem tud $E(G)$ elemszáma felé nőni.
\end{bizonyitas}

\begin{tetel}{EULER-ÚT LÉTEZÉSE}
Egy összefüggő G gráfban akkor és csak akkor létezik Euler-út, ha a páratlan fokszámú pontok száma 0 vagy 2.
\end{tetel}

\begin{bizonyitas}{}
Az előző tétel bizonyítása alapján, ebben az esetben ha 0 a páratlan fokszámú pontok száma, akkor Euler-körről is beszélhetünk, ha 2, akkor az élsorozat nem zárt, a két végpontnak lesz eltérő a fokszáma, mivel ezt úgy tudjuk képezni, hogy a két végpontot összekötő élt elhagyjuk.
\end{bizonyitas}

\begin{definicio}{HAMILTON-ÚT ÉS KÖR}
Egy G gráfban Hamilton-körnek nevezünk egy kört, ha G minden pontját pontosan egyszer tartalmazza. Egy utat pedig Hamilton-útnak nevezünk, ha G minden pontját pontosan egyszer tartalmazza.
\end{definicio}

\begin{tetel}{SZÜKSÉGES FELTÉTEL A HAMILTON-KÖR (ÚT) LÉTEZÉSÉHEZ}
Ha a G gráfban létezik k olyan pont, melyeket elhagyva a gráf több mint k komponensre esik, akkor nem létezik a gráfban Hamilton-kör. Ha több mint k+1 komponensre esik, akkor nem létezik a gráfban Hamilton-út se.
\end{tetel}

\begin{bizonyitas}{}
Indirekt t.f.h. van a gráfban Hamilton-kör, ez legyen $(v_1, v_2,..., v_n)$ és legyen $(v_{i_1}, v_{i_2},...,v_{i_k})$ az a k pont, amelyet elhagyva a gráf több mint k komponensre esik. Az elhagyott pontok közötti ``ívek'' biztosan összefüggő komponenseket alkotnak. Pl. a $(v_{i_{1}+1}, v_{i_{1}+2},..., v_{i_{2}-1})$ is összefüggő lesz, hiszen két szomszédos pontja között az eredeti Hamilton-kör éle fut. Mivel épp k ilyen ívet kapunk, ezért nem lehet több komponens k-nál (kevesebb lehet, mivel különb. ívek közt futhatnak élek). U.a. bizonyítjuk útra. Ha egy Hamilton-útból elhagyunk k pontot, legfeljebb k+1 összefüggő. ív marad.
\end{bizonyitas}

\begin{tetel}{ELÉGSÉGES FELTÉTEL - ORE TÉTEL}
Ha az n pontú G egyszerű gráfban minden olyan $x,y\in V(G)$ pontpárra, amelyre ${x,y}\not\in E(G)$ (nem szomszédosak), teljesül az, hogy $d(x) + d(y) \geq n$, akkor a gráfban van Hamilton-kör.
\end{tetel}

\begin{bizonyitas}{}
Indirekt t.f.h. a gráf kielégíti a feltételt, de nincs benne Hamilton-kör. Vegyünk hozzá a gráfhoz éleket úgy, hogy továbbra se legyen benne Hamilton-kör. Ezt egészen addig csináljuk, amíg már akárhogyan is veszünk hozzá egy élet, lesz a gráfban Hamilton-kör. Az így kapott G' gráfra továbbra is teljesül a feltétel, hiszen új élek behúzásával ``rossz pontpárt'' nem lehet létrehozni. Biztosan van két olyan pont, hogy ${x,y} \not \in E(G')$. Ennek a behúzásával már lesz $G' + {x,y}$-ban Hamilton-kör, tehát G'-ben van Hamilton-út. Legyen ez $P = {z_1, z_2,...,z_n}$ ahol $z_1 = x$ és $z_n = y$.
Ha x szomszédos a P út valamely $z_k$ pontjával, akkor y nem lehet összekötve $z_{k-1}$-el, mert akkor az egy Hamilton-kört adna. Így tehát y nem lehet összekötve legalább d(x) ponttal, ezért
$$d(y) \leq n - 1 - d(x)$$
ami viszont ellentmondás, hisz ${x,y} \not\in E(G)$.
\end{bizonyitas}

\begin{tetel}{ELÉGSÉGES FELTÉTEL - DIRAC TÉTEL}
Ha egy n pontú G egyszerű gráfban minden pont foka legalább n/2, akkor a gráfban létezik Hamilton-kör.
\end{tetel}

\begin{bizonyitas}{}
Ez az előző tételből következik, hiszen ha minden pont foka legalább n/2, akkor teljesül az Ore-tétel feltétele, mivel bármely pontpárra $$d(x) + d(y) \geq n$$
\end{bizonyitas}
